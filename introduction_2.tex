n 2016, the United States House Republicans released a Blueprint for a new tax reform that was seen as a huge thread to the liberal economic order. Under the name 'A Better Way: Our Vision for a Confident America' and with a new destination based cash flow tax at its core,the Ways and Means Committee\footnote{Was ist das.. } attempted to reform the U.S. tax system. Next to several rather small reforms, the destination based cash flow tax (DBCFT) tackles two main pillars of the U.S. corporate taxation. The way U.S. firms are taxed domestically and internationally. Right now, a U.S. based company is obligated to pay taxes on its profit made domestically and internationally through an origin-based corporate profit tax. So while profits made through exports are taxed, imports can be deducted from the tax base. With the destination-based cash flow tax, a border tax adjustment would be applied on all profits. 
Now profits made by exported goods are exempted from the taxable profits, while imports are subject to them. This border tax adjustment plan is a crucial part of the reform and was widely criticized by the public and also used for political agendas. %This idea was created by Auerbach %While from its 'founders' Auerbach etc. invented mainly to prevent tax evasion, the now called 'Border Adjustment Tax' was simultaneously used by protectionists to push their agenda. But exactly this is what was broadly misunderstood by the public and political opinion.
Following the economic theory, a border adjustment implemented into a tax should be neutral. The border tax adjustment neutrality states that any effect of such an implementation is offset by exchange rates and prices, so that real economic values stay unchanged. Therefor only the 'good' effects such as tax evasion elimination would sustain and all distorting factors such as changes in the trade balance, relative consumer prices and wages would be untouched. It is pretty ironic, that the republican party is the strongest supporter of this reform, while its main protectionist motivation is not tackled (bedient). At least in the theory. But there is another broad misconception. Based on the early work of J.M. Keynes and A.D. Lerner, there is a widely believe that the border tax adjustment neutrality is achieved through an appreciation of the introducers currency. This thesis is dedicated to challenge this believe. I will show, that under the assumption that the border tax adjustment neutrality will hold, a depreciation follows the same condition and might be equally possible. Despite that in the long run monetary changes are irrelevant, it is crucial to have a vague conception on what can happen in the short run. 




\subsection{What is the subject of the study? Describe the economic/econometric problem.}




\subsection{What is the purpose of the study (working hypothesis)?}
The following paper will try to define the effects of a destination based cash flow tax on exchange rates under the assumption of border tax adjustment neutrality. We assume that the tax imposing country is the US, with the Dollar as the home currency, and its only trading partner is the E.U., with the foreign currency Euro. The destination based cash flow tax is not a newly introduced tax, but rather a reform that border adjusts the existing current corporate profit tax (CPT). I will therefor assume that the tax rate stays unchanged and solely concentrate on the effects associated with the border adjustment. \footnote{The actual blueprint and the implied destination based cash flow tax were meant to reduce the current 35\% corporate profit tax rate to a 20\% destination based cash flow tax. This would be one of the biggest tax cuts in the U.S. history. The idea was to finance the losses through the implemented border tax adjustment. We will see, that the unique position of the U.S. actually leads to a fiscal benefit with the introduction of a border tax adjustment. But it is questionable, if the estimated losses of 3.1 trillion USD would be compensated through that. For information on the Blueprint see \url{https://www.novoco.com/sites/default/files/atoms/files/ryan_a_better_way_policy_paper_062416.pdf} and for the actual loss in revenue see \url{http://www.taxpolicycenter.org/sites/default/files/alfresco/publication-pdfs/2000923-An-Analysis-of-the-House-GOP-Tax-Plan.pdf}} The idea behind the model is the following: If we assume that the border adjustment neutrality will hold, the economic equilibrium must be unchanged. Because a new import tax is obviously a distortion to the equilibrium, there must be some adjustments to presume the neutrality condition. The factors that are able to adjust are prices, wages and the prices of the currencies (exchange rates). Looking at the short run, we know that prices and wages tend to be sticky, but the broad economic literature does not give us enough evidence which prices are sticky and therefor constant. From this uncertainty we can derive that prices can be constant in two dimensions. The first dimension would be whether they are constant tax-exclusive or tax-inclusive. The second dimension is whether they are constant under the pricing-to-market\footnote{Also known as destination-currency-pricing}, in terms of the currency where the priced good is sold, or origin-currency-pricing assumption, in terms of the currency where the priced goods are from. Out of this two dimension, sixteen possible combinations of constant prices can result and only two of them support an appreciation of the home currency (dollar). Surprisingly, the appreciation scenario is the conventional wisdom if one look at all the articles that try to predict certain effects resulting from the implementation of a destination based cash flow tax.


\subsection{What is the innovation of the study?}
So how does this thesis differ from other papers? While there is a considerable amount of empirical and theoretical work done to explain the magnitude of an appreciation, constant or depreciatiating exchange rates assumption are rare. A reason for this is the lack of theoretical work regarding the mentioned dimension of constancies. The question whether prices are constant tax-inclusive or tax-exclusive is never asked and as I will show, solely this question can make the differenc between an appreciation and a depreciation.

\subsection{Provide an overview of your results.}
Here tax-inclusive raises, tax-exclusive drops 

\subsection{Outline of the paper}
After presenting important literature to all the parts(?) of this work in Section 2, Section 3 will rebuild a model from Buiter (2017) for the DBCFT and apply it to certain price constancy assumptions, that allow to make predictions whether or in what direction the exchange rate might change, in section 4. Section 5 will then discuss the results and also discuss whether the used assumption are credible. Section 6 will conclude this paper.  


So are there any implications for the exchange rate resulting from the border adjustment neutrality? 
As mentioned above, the exchange rate will offset the distorting effects of a border adjustment and the vast majority of the literature is assuming indeed an appreciation. I do not want to say all the literature, but at least all the literature I was able to revisit. The most influential in this case are of course primarily the creators of the destination based cash flow tax Auerbach et. al (2017) , but also the work from Feldstein and Krugman (1990), Barbiero et. al (2017), Costinot and Werning (2017), Linde (2017) and Feldstein (2017).
In the next chapter, I will follow the idea of Buiter (2017) and provide a model that proofs, under the assumption that the border tax adjustment neutrality holds, that same equilibrium will also be preserved through a depreciation. One might wonder why it is important, if the effects are the same. The problem are the bets of the financial markets. If all actors bet on an appreciation of the dollar for example, an actual depreciation would lead to nothing less but a massive financial chaos.
