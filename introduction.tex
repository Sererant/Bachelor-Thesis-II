

\section{Introduction}

In 2016, the House Ways and Means Committee of the United States brought forward a draft tax reform to transform the existing (origin based) corporate profit taxation into a destination based corporate profit taxation.\footnote{The Committee on Ways and Means is the chief tax-writing committee of the United States House of Representatives. \url{https://waysandmeans.house.gov}. The planned tax is also known as destination based cash flow tax or border adjustment tax.} This transformation would be achieved through a border tax adjustment. \\
Motivated by the political and economic discussions that followed this proposal, I will investigate possible implications of a border tax adjustment for exchange rates in the context of the planned destination based cash flow tax.  Economic theory largely agrees on the neutrality of an implementation of such a border tax adjustment, meaning there would only be effects on nominal, not real, economic measures. There also seems to be consensus that this neutrality would lead to a change in the nominal exchange rates, more specifically,  an appreciation of the currency of the implementing country.\footnote{In the following, whenever exchange rates are mentioned, nominal exchange rates are meant.}
My thesis will follow the work of \cite{buiter2017exchange}, who questions this prevailing opinion and shows that a neutral border tax adjustment could, under certain assumptions, also cause a depreciation. %I will therefore replicate the model put forward by \cite{buiter2017exchange} and briefly outline and discuss the implications of his results.\\ 

\subsection{Border Tax Adjustment Neutrality and Exchange Rates}
When exports are taxed and imports are exempted in a tax regime, this is called origin based taxation, which is currently the case in the U.S.. The in 2016 proposed desination-based corporate profit tax would move the U.S. from an origin based taxation to a destination based taxation, where imports are taxed and exports are exempted. This change is called a border tax adjustment. For somebody living in a country with a value-added tax this might sound familiar, given that border tax adjustments can be part of direct (corporate profit tax) and indirect (value-added tax) taxes.\footnote{Border tax adjustments can actually be part of numerous ways of taxation, such as in \cite{mckibbin2018role} for carbon taxes.}

~\\ 
At first sight this might seem like a distortion. However, the introduction of a border tax adjustment has no real effects on the economy. Nominal adjustments of prices, wages and the exchange rate would mostly offset any effects following an implementation. This is called border tax adjustment neutrality. Given the Keynesian proposition of sticky prices and wages, the neutrality must go along with changes in the exchange rate.\\
This topic is not new. Related economic literature was broadened since value-added tax regimes became more and more popular during the mid of the 20th century. Following a discussion on possible negative effects, \cite{Feldstein&Krugman} showed the neutrality of a border tax adjustment in the context of a value-added tax. And since the effects of a value-added tax are identical to those of a destination based corporate profit tax, it theoretically proofed the main assumption of my model. 

But the main focus of this thesis will not be the border tax adjustment neutrality, since I simply assume that it holds for the model and its analysis. It is rather on the implications of a border tax adjustment for the nominal exchange rates. As I will show, the economic literature broadly assumes that following an adjustment and to maintain the neutrality, the exchange rate will offset any effects through an appreciation by the same rate as the imposed tax. I will thus replicate the work of \cite{buiter2017exchange}, which shows, that one can not simply assume this, since the border tax adjustment neutrality also prevails with a depreciation. Hence, he extends the common assumption of price constancy by an additional dimension. So that not only the currency pricing matters, but also whether prices are constant tax-inclusive or -exclusive. But since there is a lack of empirical evidence, we must assume that prices are sticky in every possible combination of the following two constancy dimensions (a) and (b):

\begin{itemize}
\item[(a)] \label{const1} Prices stay constant in the currency of the producer (origin currency pricing), or in the currency of the consumer (destination currency pricing)
\item[(b)] \label{const2} Prices are either constant excluding tax or constant including tax 
\end{itemize}\\

Let us assume that the U.S. is our home country and has one trading partner, the E.U.. We also assume that prices are constant in the currencies of the producers and are constant tax-exclusive. This means that the nominal prices for imports are constant in Euro and the nominal prices for exports are constant in Dollar following the implementation of a border tax adjustment. Tax-exclusive well noticed. Under the assumption of border tax adjustment neutrality the model provided by \cite{buiter2017exchange} implies an appreciation of the nominal exchange rate by the magnitude of the imposed tax. Yet, still in the framework of \cite{buiter2017exchange}, if we change the assumption from constant tax-exclusive to constant tax-inclusive prices (while holding everything else constant econ term), a neutrality can only be met through a depreciation of the nominal exchange rate. 
After extending this thought, there are sixteen possible price constancy assumption within the two dimensions. As provided in table 1, two of them lead to an appreciation, two to a depreciation. Four are inconsistent with the border tax adjustment neutrality assumption, four are indeterminate and four lead to constant exchange rate. We will see that only the four appreciation and depreciation scenarios are feasible, thus my analysis will focus only on them. 


\subsection{Survey of relevant literature}
In the following section, I will provide a literature summary for all the important aspects of this thesis. \\

Border tax adjustments are mostly part of certain tax policies, often used in the context of valued-added taxes or in our case the destination based corporate profit tax, that was firstly introduced by \cite{avi2000globalization} and \cite{bond2002cash}. Under the name destination-based cash flow tax
It was famously picked up by the House Means and Way Committee under the leadership of Republican Speaker Paul Ryan in 2016, to reform the current corporate taxation in the U.S.. They proposed a reduction of the existing corporate profit tax rate from 35\% to 20\%, while adding a border tax adjustment. So that imports are no longer deductible from the tax base, whereas exports, the payroll and domestic intermediates are.\footnote{There are further aspects of the proposed Blueprint that differ from the original destination based cash flow tax. For example, it would allow for a full and immediate expensing of capital investment. For more information have a look here on the Border Adjustment section on page 27 and 28 in \url{https://abetterway.speaker.gov/_assets/pdf/ABetterWay-Tax-PolicyPaper.pdf}} This would mimic a combination of an uniform import tax and export subsidy and would shift the U.S. from a origin based to a destination based tax regime. According to \cite{auerbach2017destination}, this has political and economical advantages. Economically it limits the incentives for profit shifting, whereas the newly imposed import tariff is politically used to raise government revenue in order to compensate for the significant cut in the tax-rate (\url{https://voxeu.org/article/border-adjustment-tax}).\footnote{We will see, that a unique position of the U.S. actually leads to a fiscal benefit with the introduction of a border tax adjustment. But it is questionable, if the estimated losses of 3.1 trillion USD would be compensated through that. For information regarding the expected losses, see \url{http://www.taxpolicycenter.org/sites/default/files/alfresco/publication-pdfs/2000923-An-Analysis-of-the-House-GOP-Tax-Plan.pdf}.} 

As mentioned above, despite the change in the tax regime, the border tax adjustment should have no effect on real economic values. Reason for that is the border tax adjustment neutrality.
The foundation of the border tax adjustment neutrality lies in a proposition firstly presented by \cite{lerner1936symmetry}. The later called Lerner Symmetry demonstrates the symmetry between import tariffs and export taxes. He showed that indeed, an import tariff and an export tax distort trade, but equally. This result is not intuitive, since it would mean that a country can subsidies exports through subsiding imports. \footnote{See \cite{bhagwati1998lectures} for a deeper analysis of the Lerner Symmetry.}   \cite{grossman1980border} put this idea one step further by assuming, that the symmetry is also transitive: If the effect of an import tax (tariff) equals the effect of an export tax, then an import subsidy should also equal an export subsidy. So that (1) an import tax plus an import subsidy or (2) an export subsidy plus an export tax have no effect on imports, exports and other economic outcomes. This applies also to (3) an import subsidy plus an export tax or (4) an import tax plus an export subsidy, so that there is also no effect. The change from (3) to (4) is our border tax adjustment case.  \footnote{There is actually a real concern, if one can assume this transitivity. \cite{casas1991lerner} argues that \cite{lerner1936symmetry} does not imply the equivalence of an import tariff and an export subsidy. Also \cite{linde2017macroeconomic} shows that within a New Keynesian Model there are significant long term deviations from the Lerner Symmetry.}
So, in a deterministic environment and under the assumption of nominal short-term rigidity of prices and wages, a given import tariff and a given export subsidy by the same rate should lead to an exchange rate movement, that fully cancels out the real price distortions to maintain the real equilibrium allocation and therefor maintain our border tax adjustment neutrality. \\
Surprisingly J.M. Keynes already mentioned a related outcome in 1931 (\cite{macmillan1931report}). In a report to the British Parliament he wrote that a import tariff and an export subsidy by the same rate have the same effects as a devaluation of the imposing countries currency (in this case of the British pound), while the gold pound parity holds. Also for fixed exchange rates, \cite{farhi2013fiscal} showed that a newly introduced value added tax, in this case as a replacement for the employers share of the payroll tax, would also mimic the same effects as a devaluation of the imposing countries currency. Under the assumption that exchange rates are flexible \cite{meade1974note} noted, that either the home currency must appreciate or there must be an inflation of nominal prices to offset this devaluation and therefor maintain the Lerner Symmetry and the implying border tax adjustment neutrality. This supports at least one of our outcomes \eqref{i}. The Lerner Symmetry  was later translated by \cite{mckinnon1966intermediate} and \cite{grossman1980border} to a more general model with multiple goods and by \cite{Feldstein&Krugman} for the implementation of a value added tax. \footnote{The latter proof is essential for us. We will later consider, that under certain circumstances, a value added tax as equal to the corporate profit tax and therefor, in our case, equal to the destination based cash flow tax. So that we can assume the neutrality will hold.} While \cite{lerner1936symmetry} built his model for the steady state with competitive markets, balanced trade and constant returns to scale, \cite{Feldstein&Krugman} also showed that the Lerner Symmetry is still accountable over multiple periods and inter-temporally balanced trade. Further, \cite{costinot2017lerner} found out that the Lerner Symmetry only holds if one assumes that production technology is not relocatable and consumers can not move between countries. I will discuss the feasibility of the Lerner Symmetry and the Border tax adjustment neutrality more deeply in section \ref{discussion}.\\
As mentioned above, we are considering that prices can be either constant in the terms of the origin or the destination country.
If prices are constant in terms of the origin countries currency, it is called 'origin-currency-pricing' or 'producer-currency-pricing'. When it is constant in terms of the destination countries currency, it is called 'destination-currency-pricing', 'pricing-to-market' or 'local-currency-pricing'. The names vary across the literature. The history of destination currency pricing comes from the U.S. during 1980's, when, to the surprise of consumers and economists, prices of imported goods did not fall despite a strong appreciation of the dollar. The term 'pricing-to-market' was then firstly introduced by \cite{krugman1986pricing} and some empirical papers are also supporting this assumption, with the most influential work, next to \cite{krugman1986pricing}, done by \cite{gil2002export} and \cite{marston1990pricing}. \cite{gil2002export} looked at data corresponding to pricing-to-market behavior in European car markets between 1993 and 1998. For imported cars, he found that destination currency prices are constant. \cite{marston1990pricing} investigated pricing-to-market behavior of Japanese firms. While Japan is in an unusual position of a constant depreciation, he discovered that product prices of exporting firms varies drastically relative to the price of the same products produced for the domestic market. The 'competitor' is the origin currency pricing assumption. So relative export to domestic prices of a given country are constant following an exchange rate movement. So in the end, there is no evidence for one or another. The second dimension, whether prices are constant tax-exclusive or tax-inclusive, is even weaker. As \cite{buiter2017exchange} mentions, there is little to no literature on this topic. He simply refers to the tax-exclusive as the 'Keynesian Textbook Example'. \\ 
