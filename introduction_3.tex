

\section{Introduction}
The following thesis concerns the implications of a border tax adjustment on exchange rates. Economic theory largely agrees on the neutrality of an implementation of such a border adjustment and therefor has only effects on nominal and not real economic measures. There also seems to be a consensus, that this neutrality will be achieved through a change in the nominal exchange rate, more specific through an appreciation of the currency of the implementing country. \cite{buiter2017exchange} questions this consensus, and shows, that a depreciation is very well possible as a result of border tax adjustment. In this thesis, I will replicate \cite{buiter2017exchange}, explaining the given assumptions and his model, and briefly outline implications of his finding.  

%The econimic effects of such a border tax adjustment recieved growing interest, since in 2016, the House Mean and Ways committee of the Republican Party released a blueprint to change the U.S. taxation system. It picked up the idea of a destination-based cash flow tax from Avi-Yonah (2000) and Bond and Devereux (2002), that was mainly pushed   

Is a country operating a tax regime, that taxes exports and exempts imports, this is called origin based taxation. Let us take the U.S. as an example, because its current corporate profit taxation is satisfying those criteria. Would, as planned in 2016, this corporate profit tax be replaced by a border adjusted corporate profit tax, the taxation of commodities switches and imports are subject to the tax, while exports are exempted. The U.S. would change into a destination based tax regime and this change is called a border tax adjustment. For one living in a country with a value-added tax, this sounds somewhat familiar. "since the border adjusting effects of a value-added tax are identical to the one of a destination based corporate profit tax."

%The macroeconomic effect of such a border tax adjustment reveived growing interest, since the U.S. House Republicans included 

At first sight this seems like a distortion, but, at least in theory, it actually has no real effects on the economy. Nominal adjustments of prices, wages and the exchange rate would offset any effects following the implementation. This is called border tax adjustment neutrality and because prices and wages are assumed to stay sticky in the short-run, the neutrality must be achieved through a change in the exchange rate.

%This topic is not new. Since the value-added tax became more and more popular in the mid of the 20th century, the economic literature was broadened on this topic. Following a discussion on possible negative effects, \cite{Feldstein&Krugman} delivered a proof for the border tax adjustment neutrality in the case of a value added tax. 

%This topic first came up, when value-added taxes became popular in mid-20th century. The implementation of a value-added tax has the same consequences as an introduction of a destination-based corporate profit tax (…). Answering to a political debate about possible negative effects of value-added taxes, Feldstein and Kurgman published a paper in 1990 (…), which proved border tax adjuments neutrality. 

%But the focus of this thesis is not going to be the border tax adjustment neutrality, since I will simply assume that it holds. It is rather on its implications for exchange rates, because there is a wide spread believe, t

%For this thesis, I will not focus on border tax adjustment neutrality, but rather on its implications for exchange rates. As mentioned, economic literature usually assumes, that the exchange rate adjusts for border tax neutrality to be given. I will follow Buiter (2017) in showing, that this is not given without further assumptions, and that neutrality can also be respected when the currency, in my case the dollar, depreciates or stays constant. (verstehe das DENN in dem Originalsatz nicht ganz)

My thesis will thus replicate \cite{buiter2017exchange}, who shows exactly that. He describes that price constancy assumptions are essential  There are two price constancy assumptions he uses: 

\begin{itemize}
\item[(a)] \label{const1} Prices stay constant in the currency of the producer (origin currency pricing), or in the currency of the consumer (destination currency pricing)
\item[(b)] \label{const2} Prices are either constant excluding tax or constant including tax 
\end{itemize}

Looking at the short run, we know that prices tend to be sticky. But the broad economic literature does not have enough evidence say which one is sticky and therefor may stay constant. From this uncertainty we can derive that prices can be constant in two dimensions, \ref{const1} and \ref{const2}.

To get a feeling of tax adjustments, let’s look at the United States and its trade partner, the EU. Assuming prices are constant in the currency of the producer, excluding tax. This means, that after the replacement of an origin based tax with a destination based tax, the prices, after tax, for imports in euro and the prices, after tax, for exports in dollar have to be same as before the border tax adjustment. In Buiter’s model (2017), this can only be achieved, if the dollar appreciates by the same amount as the collected tax. (in percentage points? und: ich dachte buiter sagt genau dass es auch anders geht?)

Yet, if the assumption is, that prices stay constant including taxes, and in the currency of the producer, neutrality can only be achieve by a depreciation of the dollar by the amount of the collected tax. (in percentage points, oder wie auch immer)

Given the two dimensions of price constancy, 16 possible price constancy combinations emerge. Two include an appreciation, two a depreciation, four achieve a constant exchange rate (though assuming constant prices both tax inclusive and tax exclusive, thus I will ommit from my analysis), four violate neutrality (warum sind die dnan bei den 16 dabei überhaupt? bzw, warum schauen wir uns denn die an die price neutrality machen, nicht border tax adjument neutrality?) and four are not defined (what does that mean?). In this thesis, I will focus only on the scenarios that I find most likely: the appreciation and despreciation scenarios. 

\subsection{Intro Deutsch}
Die folgende Arbeit wird sich mit den Auswirkungen eines steuerlichen Grenzausgleich auf Wechselkurse befassen. Die ökonomische Theorie ist sich weitestgehend darüber einig, dass die Implementierung eines solchen Ausgleiches keine Folgen fuer Realökonomische Grössen hat, also neutral ist. Sie scheint sich auch darüber einig zu sein, dass diese Neutralität durch eine Aufwertung der Währung des einfuehrenden Staates geschieht. 
Meine Thesis folgt der Arbeit von Buiter(2017), der die Annahme der Aufwertung in Frage stellt und zeigt, dass auch eine Abwertung die Neutralitätskriterien erfüllt. 

In 2016, the House Ways and Means Commitee of the United states attempted a tax reform that would, that would border adjusts the existing corporate profit taxation. \footnote{The Committee on Ways and Means is the chief tax-writing committee of the United States House of Representatives. \url{https://en.wikipedia.org/wiki/United_States_House_Committee_on_Ways_and_Means}} Motivated by the followed political and economic discussion about this decision, I will investigate possible exchange rate implication of such a border tax adjustment on exchange rates in the context of the planned destination based cash flow tax.  Economic theory largely agrees on the neutrality of an implementation of such a border adjustment, meaning there are only effects on nominal and not real economic measures. There also seems to be a consensus, that this neutrality will be achieved through a change in the nominal exchange rate, more specific through an appreciation of the currency of the implementing country. My thesis will follow the work of 
\cite{buiter2017exchange}, that questions this consensus and shows, that also a depreciation is satifying the neutrality conditions. I will therefor replicate \cite{buiter2017exchange}, explaining the given assumptions and his model, and briefly outline and discuss the implications of his finding.  

The following thesis studies the theoretical framework surrounding the introduction of a border adjustment tax regarding exchange rates. Economic theory overwhelmingly states that the implementation of such a regime only affects economic variables in nominal terms, whereas real variables such as exchange rates or real wages stay neutral. Furthermore, it largely agrees that this neutrality is based on a nominal appreciation of the currency of the implementing country. \textbf{citation?} However, this thesis follows the recent paper by \cite{buiter2017exchange}, which challenges the exclusive assumption of a nominal appreciation and shows that the neutrality of real variables following a border adjustment tax can also be achieved through a depricaition. 

Befindet sich ein Staat in einem Tax Regime, das Exporte besteuert und Importe von dieser Steuer ausnimmt, so nennt man das origin based taxation. Nehmen wir als Beispiel die USA. Die dort existierende CPT erfüllt eben diese Kriterien. Wird jedoch, wie 2016 angedacht, diese existierende Unternehmenssteuer durch eine grenzausgleichende Steuer ersetzt, so werden fort an Exporte von der Steuer befreit und Importe mit selbiger Steuern belegt. Die USA wuerde somit in ein destination based tax regime wechseln. 
Diesen Wechsel nennt man Border Tax Adjustment, also steuerlichen Grenzausgleich.

Is a country operating a tax regime, that taxes exports and exempts imports, this is called an origin based taxation. Let us take the U.S. as an example, since its current corporate profit taxation is satisfying those criteria. Would, as planned in 2016, this corporate profit tax be replaced by a border adjusted corporate profit tax, the taxation of commodities switches and now imports are subject to the tax, while exports are exempted. The U.S. would therefor change into a destination based tax regime and this change is called a border tax adjustment. For one living in a country with a value-added tax, this might sounds somewhat familiar, since its effects are identical to those of a destination based corporate profit tax. 

Border tax systems can generally be divided into origin - and destination based tax regimes. The former signifies a taxation of only exported goods and services whereas the latter imposes a tax purely on imports. The current tax regime of the U.S., the CPT, is an example of such an origin based tax. 

Was zunächst nach einem grossen Einschnitt, fast schon protektionistischen Einschnitt aussieht, hat jedoch theoretisch keine Auswirkungen auf eine Volkswirtschaft. Denn es wird davon ausgegangen, dass durch der Implementierung folgenden Preis-, Lohn- und Wechselkursänderungen jegliche Effekte auf die Wirtschaft weitestgehend eliminiert werden. Diese Annahme nennt man BTA neutrality. Und da man annimmt, dass zumindest im Short Run Preise und Löhne sticky sind, so muss die Neutralität durch eine Änderung im Wechselkurs geschehen. 


At first sight this seems like a distortion, but, at least in theory, it actually has no real effects on the economy. Nominal adjustments of prices, wages and the exchange rate would mostly offset any effects following the implementation. This assumption is called border tax adjustment neutrality and because of the Keynesian proposition of sticky prices and wages, the neutrality must be achieved through a change in the exchange rate.

Diese Thema ist nicht neu. Seitdem sich seit Mitte des Jahrhunderts die Mehrwertsteuer immer größerer Beliebtheit erfreut, mehrt sich auch die Literatur zu diesem Thema. Denn die Einfuerhung einer Mehrwertsteuer hat die selben, oben genannten grenzausgleichenden Effekte, wie die Einfuerhung einer destination-based corporate profit tax. Nachdem es einige politische Diskussionen über etwaige negative Effekte gab, veroeffentlichten Feldstein and Krugman ein Paper im Jahre 1990, das die Border Tax Adjustment Neutralität bewies. 

This topic is not new. Since the value-added tax became more and more popular during the mid of the 20th century, the economic literature was broadened on this topic. Following a discussion on possible negative effects, \cite{Feldstein&Krugman} showed the  neutrality of a border tax adjustment in the context of a value-added tax. And since the effects of a value-added tax are identical to those of a destination based corporate profit tax, it theoretically proofed the main assumption of my model. 


Aber das Thema dieser Arbeit wird nicht die Border Tax Adjustment Neutrality sein, denn ich werde deren Gültigkeit zunächst annehmen. Es geht vielmehr um deren Implikationen für den Wechselkurs. Denn die allgemeine volkswirtschaftliche Literatur und auch kursierende Presseberichte, nehmen an, dass die Neutralitätsbedingung durch eine Aufwertung der heimischen Währung erfuellt wird. In diesem Fall des Dollars. Wie ich zeigen werde, kann man dies nicht ohne weiteres annehmen, denn die Neutralitaetsbedingung wird auch erfuellt indem der Dollar abwertet oder konstant bleibt. 

But the main focus of this thesis will not be the border tax adjustment neutrality, since I simply assume that it holds for the model and its analysis. It is rather on the implication of the border tax adjustment neutrality for the nominal exchange rates. As I will show, the economic literature broadly assumes that following an adjustment and to maintain the neutrality, the exchange rate will offset any effects through an appreciation by the same rate as the imposed tax. I will thus replicate the work of \cite{buiter2017exchange}, which shows, that one can not simply assume this, since the border tax adjustment neutrality also prevails with a depreciation.
Hence, he extends the common assumption of price constancy by an additional dimension. So that not only the currency pricing matters, but also whether prices are constant tax-inclusive or -exclusive. But since there is a lack of empirical evidence, we must assume that prices are sticky in every possible combination of the following two constancy dimensions \ref{const1} and \ref{const2}:

\begin{itemize}
\item[(a)] \label{const1} Prices stay constant in the currency of the producer (origin currency pricing), or in the currency of the consumer (destination currency pricing)
\item[(b)] \label{const2} Prices are either constant excluding tax or constant including tax 
\end{itemize}

Looking at the short run, \cite{buiter2017exchange} assumes that prices tend to be sticky, but since there is a lack of empirical evidence, we must assume that prices are sticky in every possible combination of the following two constancy dimesnions:

But the broad economic literature does not have enough evidence say which one is sticky and therefor may stay constant. From this uncertainty we can derive that prices can be constant in two dimensions, \ref{const1} and \ref{const2}.

Nehmen wir z.b. an die USA hat einen Handelspartner, die EU. Wir nehmen auch an, dass die preise sowohl in der Währung des Produzenten, als auch tax-exclusive konstant sind. Dies bedeutend nun, dass auch nach der Einführung einer destination based tax, sich die Preise für Importe in Euro und die Preise für Exporte in Dollar auf dem selben Niveau wie vor dem border tax adjustment befinden müssen. Nach Steuern wohl bemerkt. 
Anhand des Models von Buiter (2017) kann dies nur erreicht werden, wenn sich der Dollar im gleichen Umfang wie die erhobene Steuer aufwertet. 

Ändert man nun die Annahme, dass die Preise nicht tax-exclusive, sondern tax-inclusive konstant bleiben, so kann dies nur durch eine Abwertung des Dollars im selben Umfang wie die Steuer erfolgen. 
Am Ende ergeben sich aus den zwei Dimensionen und dazugehörigen Annahmen, 16 verschiedenen Möglichkeiten, wie Preise konstant bleiben können. Zwei Kombinationen werden eine Aufwertung zur Folge haben, zwei eine Abwertung. Vier führen zu einem konstanten Wechselkurs, vier sind nicht konsistent mit der Neutralitätsbedingung und vier sind nicht bestimmt. Weil auch die vier konstanten ER Kombinationen nicht wirklich überzeugend sind, da sie voraussetzen dass Preise sowohl tax-tax-inclusive als auch tax-exclusive constant sind, werde ich mich in der anschließenden Analyse nur auf die Appreciation und Depreciation Szenarien konzentrieren. 


In the following section, I will provide a literature summary for all the important aspects of this thesis. \\

Border tax adjustments are mostly part of certain tax policies, often used in the context of the valued added tax (VAT) or in our case the destination based cash flow tax (DBCFT). The destination based cash flow tax is a corporate profit taxation system, firstly introduced by Avi-Yonah (2000) and Bond and Devereux (2002)
It was famously picked up by the House Means and Way Committee under the leadership of Republican Speaker Paul Ryan in 2016 to reform the current corporate taxation in the U.S.. They proposed a reduction of the existing corporate profit tax rate from 35\% to 20\%, while adding a border tax adjustment. So that imports are no longer deductible from the tax base, whereas exports, the payroll and domestic intermediates are.\footnote{There are further aspects of the proposed Blueprint that differ from the original destination based cash flow tax. For example, it would allow for a full and immediate expensing of capital investment. For more information have a look here on the Border Adjustment section on page 27 and 28 in \url{https://abetterway.speaker.gov/_assets/pdf/ABetterWay-Tax-PolicyPaper.pdf}} This would mimic a combination of an uniform import tax and export subsidy and would shift the U.S. from a origin based to a destination based tax regime. According to \cite{auerbach2017destination}, this has political and economical advantages. Economically it limits the incentives for profit shifting, whereas the newly imposed import tariff is politically used to raise government revenue in order to compensate for the significant cut in the tax-rate (\url{https://voxeu.org/article/border-adjustment-tax}).\footnote{We will see, that a unique position of the U.S. actually leads to a fiscal benefit with the introduction of a border tax adjustment. But it is questionable, if the estimated losses of 3.1 trillion USD would be compensated through that. For information regarding the expected losses, see \url{http://www.taxpolicycenter.org/sites/default/files/alfresco/publication-pdfs/2000923-An-Analysis-of-the-House-GOP-Tax-Plan.pdf}.} 



%It applies taxes or tax reduction to payments for goods and services crossing international borders. More specific, there is a border tax adjustment, if an origin-based tax regime, that effectively taxes exports and subsidies imports, changes to a destination-based tax regime, that effectively subsidies exports and taxes imports. 
%So for someone living in a country with a value-added tax, this sounds familiar.  Because a value added tax is, like the destination based cash flow tax,  designed to be a consumption tax. Firms are able to receive tax refunds on exported goods, as they are not consumed inside the country. A difference between the two taxes is that while imports and domestic products have the same value-added tax, domestic firms are able to deduct the payroll. As mentioned above, despite the change in the tax regimes, the border tax adjustment should have no effect on real economic values. Reason for that is the border tax adjustment neutrality.
The foundation of the border tax adjustment neutrality lies in a proposition firstly presented by \cite{lerner1936symmetry}. The later called Lerner Symmetry demonstrates the symmetry between import tariff and export taxes. He showed that indeed, an import tariff and an export tax distorts trading, but equally. This result is not intuitive, since it would mean that a country can subsidies exports through subsiding imports. \footnote{Maybe proof in appendix \cite{bhagwati1998lectures}}   \cite{grossman1980border} put it one step further by assuming, that the symmetry is also transitive: If the effect of an import tax (tariff) equals the effect of an export tax, then an import subsidy should also equal an export subsidy. So that (1) an import tax plus an import subsidy or (2) an export subsidy plus an export tax have no effect on imports, exports and other economic outcomes. This applies also to (3) an import subsidy plus an export tax or (4) an import tax plus an export subsidy, so that there is also no effect. The change from (3) to (4) is our border tax adjustment case.  \footnote{There is actually a real concern, if one can assume this transitivity. \cite{casas1991lerner} argues that \cite{lerner1936symmetry} does not imply the equivalence of an import tariff and an export subsidy. Also \cite{linde2017macroeconomic} shows that within a New Keynesian Model there are significant long term deviations from the Lerner Symmetry.}
So, in a deterministic environment a given import tariff and a given export subsidy by the same rate, should lead to an exchange rate movement that fully cancels out the real price distortions to maintain the real equilibrium allocation and therefor maintain our border tax adjustment neutrality. \\
%diese Stelle evtl. an den Anfang?%
Surprisingly J.M. Keynes already mentioned a related outcome in 1931 (\cite{macmillan1931report}). In a report to the British Parliament he wrote that a import tariff and an export subsidy by the same rate have the same effects as a devaluation of the imposing countries currency (in this case of the British pound), while the gold pound parity holds. Also for fixed exchange rates, \cite{farhi2013fiscal} showed that a newly introduced value added tax, in this case as a replacement for the employers share of the payroll tax, would also mimic the same effects as a devaluation of the imposing countries currency. Under the assumption that exchange rates are flexible \cite{meade1974note} noted, that either the home currency must appreciate or there must be an inflation of nominal prices to offset this devaluation and therefor maintain the Lerner Symmetry and the implying border tax adjustment neutrality. This supports at least one of our outcomes \eqref{i}. The Lerner Symmetry  was later translated by \cite{mckinnon1966intermediate} and \cite{grossman1980border} to a more general model with multiple goods and by \cite{Feldstein&Krugman} for the implementation of a value added tax. \footnote{The latter proof is essential for us. We will later consider, that under certain circumstances, a value added tax as equal to the corporate profit tax and therefor, in our case, equal to the destination based cash flow tax. So that we can assume the neutrality will hold.} While \cite{lerner1936symmetry} built his model for the steady state with competitive markets, balanced trade and constant returns to scale, \cite{Feldstein&Krugman} also showed that the Lerner Symmetry is still accountable over multiple periods and inter-temporally balanced trade. Further, \cite{costinot2017lerner} found out that the Lerner Symmetry only holds if one assumes that production technology is not relocatable and consumers can not move between countries. I will discuss the feasibility of the Lerner Symmetry and the Border tax adjustment neutrality more deeply in section \ref{discussion}.\\
As mentioned above, we are considering that prices can be either constant in the terms of the origin or the destination country.
If prices are constant in terms of the origin countries currency, it is called 'origin-currency-pricing' or 'producer-currency-pricing'. When it is constant in terms of the destination countries currency, it is called 'destination-currency-pricing', 'pricing-to-market' or 'local-currency-pricing'. The names vary across the literature. The history of destination currency pricing comes from the U.S. during 1980's, when, to the surprise of consumers and economists, prices of imported goods did not fall despite a strong appreciation of the dollar. The term 'pricing-to-market' was then firstly introduced by \cite{krugman1986pricing} and some empirical papers are also supporting this assumption, with the most influential work, next to \cite{krugman1986pricing}, done by \cite{gil2002export} and \cite{marston1990pricing}. \cite{gil2002export} looked at data corresponding to pricing-to-market behavior in European car markets between 1993 and 1998. For imported cars, he found that destination currency prices are constant. \cite{marston1990pricing} investigated pricing-to-market behavior of Japanese firms. While Japan is in an unusual position of a constant depreciation, he discovered that product prices of exporting firms varies drastically relative to the price of the same products produced for the domestic market. The 'competitor' is the origin currency pricing assumption. So relative export to domestic prices of a given country are constant following an exchange rate movement. So in the end, there is no evidence for one or another. The second dimension, whether prices are constant tax-exclusive or tax-inclusive, is even weaker. As \cite{buiter2017exchange} mentions, there is little to no literature on this topic. He simply refers to the tax-exclusive as the 'Keynesian Textbook Example'. I could not find evidence for this assumption. \\ 

The motivation for this thesis comes from one specific mutuality across all of the above listed literature and recent articles on exchange rate movements following a border tax adjustment such as \cite{feldstein2017house} and \cite{auerbach2017destination}. They all assume that the border tax adjustment would be prevailed through an appreciation of the introducing countries currency. The following model will mirror \cite{buiter2017exchange}'s idea, that under certain price constancy assumptions, a depreciation might also satisfy the neutrality conditions. 

%Uebergang zu Model?

