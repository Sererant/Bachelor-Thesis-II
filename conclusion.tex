\section{Conclusions}\label{Sec:Conc}

\begin{itemize}

    \item Give a short summary of what has been done and what has been
    found.

    \item Expose results concisely.

    \item Draw conclusions about the problem studied. What are the
    implications of your findings?

    \item Point out some limitations of study (assist reader in judging validity
    of findings).

    \item Suggest issues for future research.

\end{itemize}



\subsection{Why are exchange rate important?}
Exchange rate are crucial for economies. They make a country richer or poorer, less or more competitive and can dictate quantities for imports and exports. When a country decides not to fix their currency price to another countries, they are allowed to move up and down freely. And so they do. Depending on a number of variables, it can change our daily life in a direct and indirect way. Therefor their use in political agendas is somehow logical. A lower value of the euro helps Germany to maintain their position as one of the top exporters in the world, while a strong Dollar might be used by American protectionist to maintain the dominance of the US. Exchange rates are also of huge interest for financial markets, is it as speculative assets or stable investments. Depending on the size and stability of the currencies market, such interpretations tend to lean either way.
All in all one can say, that predicting a change or the direction of a change of exchange rate is crucial for individuals, industries, markets or government. Therefor its rather surprising that the economic literature focuses more on the change itself then the direction in a specific scenario, that is connected to a world wide political phenomenon. The scenario I am talking about is a border adjusting tax/tariff/subsidy policy often associated with fulfilling protectionists ideals. So called border adjustments are used by protectionists movements to emphasize their ideal of protecting the country from outside influences, achieved through import taxes and export subsidies.
