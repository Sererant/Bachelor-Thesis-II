Following \cite{buiter2017exchange}, I build my model around the border tax adjustment neutrality. It is not necessary to provide a proof for the border tax adjustment neutrality, because this was already done by numerous authors like \cite{Feldstein&Krugman} and \cite{grossman1980border}. It is enough to take and analyze the central neutrality condition, that key economic variables are not effected by the introduction of a border tax adjustment. Hence, \cite{buiter2017exchange} provides expressions of real after tax profits, tax revenue and real prices and requires their constancy. Assuming that those neutrality conditions are satisfied I follow \cite{buiter2017exchange} in his alternative pricing assumptions which in return allows to make implications for the change of nominal prices, wages and the exchange rate.\footnote{Whenever the definition of exchange rate is used, nominal exchange rates are meant.} ~\\ Even though this thesis is a replication of his model and analysis, I will exclude some aspects of his work. As my work is motivated by the introduction of a new destination based corporate profit tax in the U.S., I will tailor the model for this case and only provide the proof for its equivalence to the value added tax in the appendix.\footnote{This is necessary because there was not made a proof for the existence of border tax neutrality in the context of a destination based corporate profit tax, since it was mainly used in the context of value-added taxation. See \cite{Feldstein&Krugman}. } In addition, capital goods and capital expenditure will not be included, mainly due to the limited scope of this paper.\footnote{According to \cite{buiter2017exchange} this has no effects on results and would just lead to an unnecessary extend of the model.} 

Wie in Buiter wird die BTA N die zentrale Annahme meines Models

Firstly, I look at the conditions under which the border tax adjustment is neutral. 

The central condition under which a border tax adjustment is neutral, is the constancy of real economic variables. 

So for a border tax adjustment 

Ich werde keinen Beweis für das Neutralitätsversprechen für BTAs oder für ein vollständiges Modell einer globalen Zwei-Länder-Wirtschaft mit Handel erbringen, da dies bereits viele Male gemacht wurde, insbesondere im klassischen Artikel von Feldstein und Krugman (1990). Stattdessen werde ich Ausdrücke für eine wichtige wirtschaftliche Beziehung geben - reale Unternehmensgewinne nach Steuern - und für eine Reihe wichtiger relativer Preise, denen private Haushalte und Unternehmen im Inland (in den USA) und im Ausland (in der Eurozone) gegenüberstehen - relative Preise, deren Beständigkeit ist eine notwendige Bedingung für BTA-Neutralität. Unter der Annahme alternativer Annahmen zur nominalen Preiskonstanz gehe ich dann davon aus, dass der tatsächliche Neutralitätsvorschlag für BTAs gilt, wodurch ich die Auswirkungen eines neutralen BTA auf den US-Dollar-Wechselkurs gegenüber dem Euro aufheben kann.

Through taking the condition,that are necessary for the bta neutrality to hold , as provided by F&K and 

By defining constancy conditions of key economic variables that are necessary under which a BTA would be neutral, 


First of all, let us once again set the scope of this thesis. Under the assumption that border tax adjustment neutrality holds, I will define possible effects of border tax adjustments on the nominal exchange rate. So in order to make implications for the nominal exchange rate, I firstly have to define key economic variables and relations that must be unchanged. After defining them for the origin-based and destination-based tax regime, I look at the potential difference and then conclude with the neutrality condition. Assuming that the border tax adjustment neutrality holds, I follow \cite{buiter2017exchange} in his alternative pricing assumptions which allows then to make implications for the nominal exchange rate.


In this section I mirror the model provided by \cite{buiter2017exchange}. 

Before I start building the model, I define the sections that I will replicate from \cite{buiter2017exchange}. 


In general, he built his model for border tax adjustment following a value added tax and a corporate profit tax. Even though I provide proof for its equality in the appendix, I do not integrate the value added tax case in my model. 
Further, because we are more interested in the border tax adjustment, then in the cash flow component, I will also follow his approach to leave out capital goods and capital expenditures. As shown by \cite{barbiero2018macroeconomics} and \cite{gopinath2017macroeconomic} this is indeed correct for a general analysis and prediction, but must be adjusted if one wants to have an exact model. I will also use the possible price constancy assumption, as derived by \cite{buiter2017exchange}.  \\


First of all, let us once again set the scope of this thesis. Under the assumption that border tax adjustment neutrality holds, I will define possible effects of border tax adjustments on the nominal exchange rate. While the common perception forecasts an appreciation to offset all distorting effects, I will show that this can also be achieved through a depreciation. So in order to make implications for the nominal exchange rate, I firstly have to define key economic variables and relations that must be unchanged. After defining them for the origin-based and destination-based tax regime, I look at the potential difference and then conclude with the neutrality condition. Assuming that the border tax adjustment neutrality holds, 
I follow \cite{buiter2017exchange} in his alternative pricing assumptions which allows then to make implications for the nominal exchange rate.  