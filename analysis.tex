\section{Analysis}
Let us remember the two statements from section 1. Tax-exclusive relative prices of imports to exports fall and tax-inclusive relative prices of imports to exports rise by the tax-rate $\theta$. Why is that? Import prices are consumer prices and consumers are interested in tax-inclusive prices. As we can see in \eqref{P.M}, the neutrality condition requires them to stay constant. The tax-exclusive price of imports on the other hand is required to fall by the tax rate. 

First of all we want to define what real implications the neutrality condition has for relevant real economic values. For a border adjustment to be neutral, those value have to remain constant. The actors in an economy are producers, consumers and the government. So if we take the profit function of a representative firm, the tax revenue and relative prices and wages for consumers and producers, we are able to display all changes that might effect the actors. 
So how can we define prices that allow us to apply the two dimension of price constancy? In our case, three different prices are relevant: price of imports $P_m$, price of exports $P_x$ and the price of domestically produced and non-traded goods $P_d$. 




Before we start analyzing possible exchange rate implications, it is helpful to have a look on the BTA conditions itself. In \eqref{P.M} the consumer price neutrality condition is given and therefor the tax-inclusive relative price of imports to domestic products in the destination-based tax regime must equal those in the origin system. \footnote{The consumer price is the same for firms \eqref{mpro.o} and households \eqref{mcon.o}} Further, given the definition in \eqref{D.Tax}, the tax-exclusive price of imports to domestic goods must fall by the same percentage as the tax rate in the destination-based tax regime: 



\begin{equation}\label{mtod}
\begin{aligned}
\frac{\underline P_m}{\underline P_d} &= \frac{\overline P_m}{\overline P_d} \quad and \quad \frac{\underline e \underline P_m^*}{\underline P_d} =  \left( \frac{1}{1+\theta} \right) \frac{\overline e \overline P^*_m}{\overline P_d}
\end{aligned}
\end{equation}

One might notice that there is no more tax rate attached to the domestic prices. We assume that they are constant and can shorten the equation accordingly. If one now assumes that tax-exclusive import prices are constant in Euro and domestic prices are constant as well, this could be achieved by an exchange rate appreciation by the tax-rate $\theta$:

\begin{equation}
\begin{aligned}
\frac{\underline e \underline P_m^*}{\underline P_d} =\left( \frac{1}{1+\theta} \right) \frac{\overline e \overline P^*_m}{\overline P_d} \Rightarrow \frac{\underline e}{\overline e} = \left( \frac{1}{1+\theta} \right) \frac{\underline P_d\overline P^*_m}{\underline P_m^*\overline P_m} \Rightarrow \frac{\underline e}{\overline e} = \left( \frac{1}{1+\theta} \right) \nonumber
\end{aligned}
\end{equation}

Equal to the consumer prices, the producer price neutrality is given through the exports in \eqref{P.X} and the definition in \eqref{D.Tax}. We do not include the domestic products, because they are unaffected by the border adjustment. Domestic prices are only used to display relative changes. So as stated in Section 1, the tax-inclusive prices in the destination-based tax regime fall by the same percent as the tax rate and the tax-exclusive price is required to be unchanged:

\begin{equation}\label{xtod}
\begin{aligned}
\frac{\underline P_x}{\underline P_d} &=  \left( \frac{1}{1+\theta} \right) \frac{\overline P_x}{\overline P_d} \quad or \quad \frac{\underline e \underline P_x^*}{\underline P_d} = \frac{\overline e \overline P^*_x}{\overline P_d} 
\end{aligned}
\end{equation}

Combining both equations, we can derive the relative price of imports to exports, which leads us back to (i) and (ii) from the beginning of this chapter. 
The tax-inclusive relative price of imports to exports rises by the same percentage as the tax rate and the tax-exclusive price of imports to exports falls by the same percentage as the tax rate: 

\begin{align}
    \frac{\underline P_m}{\underline P_x} &= (1+\theta)\frac{\overline P_m}{\overline P_x} \label{mtox.1} \\
    &and \nonumber \\
    \frac{\underline P_m^*}{\underline P_x^*} &= \left(\frac{1}{1+\theta}\right)\frac{\overline P_m^*}{\overline P_x^*} \label{mtox.2}
\end{align}

\eqref{mtox.1} can also be expressed as $\frac{\underline e \underline P_m^*}{\underline P_x} = \frac{\overline e \overline P_m^*}{\overline P_x}$ and \eqref{mtox.2} as $\frac{\underline P_m}{\underline e \underline P_x^*} = \frac{\overline P_m}{\overline e \overline P_x^*}$. So the ratio of tax-exclusive import prices in dollar to tax-inclusive export prices and the ratio of tax-inclusive import prices to tax exclusive export prices stay constant respectively. This statement is not as natural as the equations above, but its important while thinking about possible constancy configurations.

\subsection{Possible Price Constancy Configurations}
I mentioned previously that there are two dimensions of price constancy with a binary outcome. Prices could be constant tax-exclusive or tax-inclusive, in Dollar or Euro. Supposing every combination is possible, it leaves us with 16 different combinations displayed in Table 1. 


\begin{center}
 \begin{table}[H]
     \centering
     \caption{Exchange Rate Responds to different price constancy settings}
     \begin{tabular}{|c| c| c| c |c|} 
     \hline
      & X in \$ & X in \$ & X in \euro{} & X in \euro{} \\
      & tax-exclusive & tax-inclusive & tax-exclusive & tax-inclusive \\
     \hline
     M in \euro{} & appreciation & constant & inconsistent & indeterminate \\
     Tax-exclusive & (i) & (ii) & (iii) & (iv) \\
     \hline
     M in \euro{} & constant & depreciation & indeterminate & inconsistent \\
     Tax-inclusive & (v) & (vi) & (vii) & (viii) \\
     \hline
     M in \$ & inconsistent & indeterminate & depreciation & constant \\
     Tax-exclusive & (ix) & (x) & (xi) & (xii) \\
     \hline
     M in \$ & indeterminate & inconsistent & constant & appreciation \\
     Tax-inclusive & (xiii) & (xiv) & (xv) & (xvi) \\ 
     \hline
    \end{tabular}
     
     \label{tab:my_label}
 \end{table}
\end{center}

So apparently, out of the sixteen possible combinations, two imply an appreciation and two a depreciation of the exchange rate. \footnote{It is notable for the following, that we measure the nominal exchange rate dollar per euro. So a decrease in e is an appreciation (the dollar gets more valuable compared to the euro).} Four results in an unchanged exchange rate, four are inconsistent with the BTA neutrality itself and four are indeterminate. 
I will conclude, that only the four scenarios describing an appreciation and depreciation are considerable.

The model used the neutrality condition from after profit tax, tax revenue and the relative prices to derive equations \eqref{mtod} - \eqref{mtox.2}. We are now using this neutrality conditions and the definitions from \eqref{O.Tax} and \eqref{D.Tax} to see the effect of different price constancy assumptions from table 1. Afterwards I analyze the effects on the relative economic variables from section Model, by evaluating their effect through looking on the differences. 

\subsubsection{Appreciation Scenario}
The appreciation scenario is supported by two constant price configurations: (i) a tax-exclusive origin currency pricing and (xvi) a tax-inclusive pricing to market combination. 
\subsubsection*{(i)}
\begin{equation}\label{i}
\begin{aligned}
&\text{Constant tax-exclusive import prices in euro:}\\ &\underline P^*_m = \overline P^*_m \quad or \quad \frac{\underline P_m}{(1+\theta)\underline e} = \frac{\overline P_m}{\overline e} \\
&\text{and}\\
&\text{Constant tax-exclusive export prices in dollar:}\\  &\underline e \underline P^*_x = \overline e \overline P^*_x \quad or \quad \underline P_x = \frac{\overline P_x}{1+\theta}
\end{aligned}
\end{equation}

Let us begin with the constant tax-exclusive import price in dollar $P_m^*$. The border adjustment neutrality condition for tax-exclusive imports to exports in \eqref{mtox.2} implies, that while $P_m^*$ is constant, the tax-exclusive price for exports in euro must rise in the destination regime by the tax-rate $ \underline P^*_x = (1+\theta) \overline P^*_x$. Since the second condition is a constant tax-exclusive export price in dollar $e  P^*_x$, it could only be achieved by an appreciation of the dollar by the given tax-rate $\theta \Rightarrow$ $\frac{\underline e}{\overline e} = \frac{1}{(1+\theta)}$. \footnote{One could achieve the same results by starting with the second condition of constant $e  P^*_x$. The alternative neutrality condition of \eqref{i} then implies a constant import price in dollar $P_m$. Now the first line of \eqref{i} gives us the following: A constant $P_m$ is then $\frac{\underline P_m}{\overline P_m} = 1$ so that  $\frac{\underline P_m}{\overline P_m} = \frac{(1+\theta)\underline e}{\overline e} = 1 \Rightarrow \frac{\underline e}{\overline e} = \frac{1}{(1+\theta)}$}
\\
With $eP^*_x$ constant, we know that $P_d$ is constant as well in \eqref{xtod}. It follows, that the wage $w$ is constant in the US and in Europe (42) \footnote{as the difference is 0}, the US real consumption wage (30) , the real US product wage in all sectors (39, 43) and the price of domestic goods in the EU are also constant. \\
This configuration is, as mentioned above, a origin currency pricing and therefor the often used example, especially for Keynesian economists. Following the Keynesian theory of sticky prices, in the short run there shouldn't be any changes in the prices. And indeed, as described above all the prices and wages stay constant. But two prices are missing. Price rigidities must be consistent over the borders. Meaning, when tax-exclusive prices of imports are constant in dollar, they also must be constant in euro. This is actually the case. While the tax-inclusive price of imports in euro rises by the tax-rate, through the appreciation of the dollar, the tax-exclusive price will be equal in dollar. So that $\frac{\underline P_m}{(1+\theta)\underline e} = \frac{\overline P_m}{\overline e}$ with definition (1) and (2) for $\underline P$ and $\overline P$:\newline $\frac{(1+\theta)\underline e \underline P_m^*}{(1+\theta)\underline e} = \frac{\overline e \overline P_m^*}{\overline e}$ \Rightarrow \quad $\underline P_m^* = \overline P_m^*$. Therefor all relevant prices are sticky, so we neither have a inflation or deflation. It might be the reason, why this scenario is the 'most popular' under economists and would obviously be J.M. Keynes first choice. 

\subsubsection*{(xvi)}
\begin{equation}\label{xvi}
\begin{aligned}
&\text{Constant tax-inclusive import prices in dollar:}\\ &\underline P_m = \overline P_m \quad or \quad (1+\theta)\underline e \underline P^*_m  = \overline e \overline P^*_m\\
&\text{and}\\
&\text{Constant tax-inclusive export prices in euro:}\\ &\frac{\underline P_x}{\underline e } = \frac{\overline P_x}{\overline e} \quad or \quad  \underline P^*_x = (1+\theta)\overline P^*_x 
\end{aligned}
\end{equation}
Similar to (i), but now as a condition, $P_m$ is constant. From \eqref{mtod} we know that if $\underline P_m = \overline P_m$, then $\frac{\underline P_m}{\overline P_m} = 1 = (1+\theta)\frac{\underline P_x}{\overline P_x} \Rightarrow \frac{1}{(1+\theta)} = \frac{\underline P_x}{\overline P_x}$ so that $\underline P_x = \frac{\overline P_x}{(1+\theta)}$. In other words, the tax-inclusive price for exports must fall by same magnitude as $\theta$. Now the second line of \eqref{xvi} requires an fall of e in the destination regime and therefor as well an appreciation of the dollar: $\frac{\underline P_x}{\underline e } = \frac{\frac{\overline P_x}{(1+\theta)}}{\overline e} = \frac{\overline P_x}{\overline e (1+\theta)}$ so that $\frac{\underline e}{\overline e} = \frac{1}{(1+\theta)}$.\\
Further, with constant $P_m$, the price for domestic goods $P_d$ must be constant. Otherwise the neutrality condition for consumer prices in the US (19) wouldn't hold. An similar to (i) the wage $w$ is constant in the US (30) and in Europe (42) (as the difference is 0), the US real consumption wage (30) , the real US product wage in all sectors (35, 39) and the price of domestic goods in the EU are also constant. While also leading to an appreciation, this constant price configuration represents the rather new assumption of pricing-to-market. While backed by a decent amount of empirical evidence, its tax-inclusive or tax-exclusive nature was never part of those analysis. Nevertheless, we see in \eqref{xvi} and under consideration of \eqref{D.Tax} and \eqref{mtox.1} that while the tax-inclusive price of exports in euros is constant, the tax-exclusive price of exports in euro rises by $\theta$. Also, the tax-exclusive price of exports in dollar stays constant, while the tax-inclusive one falls by $\theta$. As in \eqref{i}z there is neither an inflation or deflation. 

\subsubsection{Depreciation}
Apparently, (i) and (xvi) from table 1 result in an appreciation of the dollar. Now we are considering the two depreciation scenarios. 

\subsubsection*{(vi)}
\begin{equation}
\begin{aligned}
&\text{Constant tax-inclusive import prices in euro:}\\ &\frac{\underline P_m}{\underline e} = \frac{\overline P_m}{\overline e} \quad or \quad \underline P^*_m  = \frac{\overline P^*_m}{(1+\theta)} \\
&\text{and}\\
&\text{Constant tax-inclusive export prices in dollar:}\\ &\underline P_x = \overline P_x \quad or \quad \underline e \underline P_x^* = (1+\theta)\overline e \overline P^*_x
\end{aligned}
\end{equation}

Once again, the two sides in line 1 result from the condition itself and the definition stated in \eqref{O.Tax} and \eqref{D.Tax}. Therefor, if we assume that the tax-inclusive prices of imports in euro stay constant, we can express this also as $\underline P^*_m  = \frac{\overline P^*_m}{(1+\theta)}$. Through \eqref{mtox.2} we know, that if $\frac{\underline P_m}{\underline e} = \frac{\overline P_m}{\overline e}$ and $\frac{\underline P_m}{\underline e \underline P_x^*} = \frac{\overline P_m}{\overline e \overline P_x^*}$ it follows that $\underline P^*_x = \overline P^*_x$. And since $\underline P_x = \overline P_x$ it follows from the alternative expression of \eqref{mtox.1} that $\underline e \underline P^*_m = \overline e \overline P_m^*$. Taking the right-hand side of line 1 in (66) we conclude that $\underline P^*_m  = \frac{\overline P^*_m}{(1+\theta)}$ in $\underline e \underline P^*_m = \overline e \overline P_m^*$ implies that $\underline e = \overline e (1+\theta)$. In other words, the dollar is going to depreciate. \\
Following this depreciation, the tax-inclusive price of exports in euro and the tax-exclusive price of imports in euro fall by $\theta$. The tax-inclusive price of imports in dollar the tax-exclusive price of exports in dollar rises by $\theta$.
Contrary to the two appreciation cases, according to \eqref{xtod}, if the $P_x$ is constant, the price of domestic goods $P_d$ has to rise by the tax rate $\theta$, as well as the US wage $w$ (31) and the US tax-inclusive price of imports $P_m$ (19). On the other hand, the EU prices of EU domestic goods $P^*_d$ (46) and the EU wage $w^*$ (42) will fall by the tax rate $\theta$. The price of exports in euro $P^*_x$ will remain constant. 
This is the origin currency pricing setting that leads to a depreciation. Even though it is the setting that would, like in (i), be assumed by Keynes, it leads to an inflation in the home country and a deflation in the foreign country. Because those are changes in the short term, it does not go well with Keynes theory of sticky prices and wages. 

\subsubsection*{(xi)}
\begin{equation}\label{xi}
\begin{aligned}
&\text{Constant tax-inclusive price of imports in dollar:}\\ &\underline e \underline P^*_m = \overline e \overline P^*_m \quad or \quad \frac{\underline P_m}{1+\theta} = \overline P_m \\
&\text{and}\\
&\text{Constant tax-exclusive price of exports in euro:}\\ &\underline P^*_x = \overline P^*_x \quad or \quad \frac{\underline P_x}{\underline e} = \frac{\overline P_x}{(1+\theta)\overline e}
\end{aligned}
\end{equation}

The right-side of line one states, that the tax-inclusive import price must rise by the tax rate $\theta$ in the destination regime. Implementing this in the BTA neutrality condition \eqref{mtox.1}, the tax-inclusive export price $P_x$ stays constant. Now the right-hand side of line 2 in \eqref{xi} requires a depreciation of the dollar by the tax-rate $\theta$: $\frac{\underline P_x}{\underline e} = \frac{\overline P_x}{(1+\theta)\overline e} \Rightarrow \frac{\underline P_x}{\overline P_x} = 1 = \frac{\underline e}{(1+\theta)\overline e} \Rightarrow \frac{\underline e}{\overline e} = (1+\theta)$.\\
So the tax-inclusive price of exports in euro will fall and the tax-inclusive price of imports in dollar will rise. Following the depreciation of the dollar, while the tax-inclusive price of exports in dollar and the tax-inclusive price of imports euro remain constant, the tax-exclusive price of exports in dollar will rise and the tax-exclusive price of imports will fall by $\theta$.
As in (vi), the price of domestic goods $P_d$ (60) and the US wage $w$ (31) will rise by the tax rate $\theta$. The only difference is, that not the export neutrality condition requires that, but the import neutrality condition in (60). When we take look at (42) and (50), the euro wage and price for domestic goods in the euro zone will fall as well. This is the depreciation with pricing-to-market. Similar to (vi), we see an inflation in the home country and a deflation in the foreign country. In terms of sympathy by Keynes, this would rank last. Neither are his pricing assumptions represented, nor would the theory of sticky prices and wages hold. \\

So in the end, out of the 16 possible price constancy combinations, two will result in an appreciation of the dollar and two in a depreciation. Before we discuss this result, I will shortly explain why we drop the other 12 combinations and provide a proof in Appendix x.

\subsubsection{Inconsistent with the BTA neutrality}
Firstly we can rule all combinations that are inconsistent with BTA neutrality, so that the assumption stated in table 1 are either contrary to \eqref{mtox.1} or \eqref{mtox.2}. (iii) describes a constant , while (vii) describes a fall by $(\theta)^2$ of the tax-exclusive relative price of imports to exports. But the BTA neutrality condition in \eqref{mtox.2} requires a fall of the tax-exclusive relative price of imports to exports by simply the tax-rate $\theta$.  \\
(xiv) and (ix) on the other hand imply the tax-inclusive relative price of imports to exports to remain constant and that the tax-inclusive relative price of imports to exports rises by $(\theta)^2$, respectively. This is not consistent with the neutrality condition in \eqref{mtox.1}, that requires the tax-inclusive relative price of imports to exports to rise by the same magnitude as the tax-rate $\theta$.

\subsubsection{Constant}
There are also four constant price configurations that support a constant nominal exchange rate. While they might be justifiable through the neutrality assumption, they lack in a logical reason. They either have import prices constant net-of-tax and export prices constant including tax or import prices constant including tax but export prices constant net-of-tax. Prices can only be sticky in one 'category'. 

\subsubsection{Indeterminate }
Lastly, (xii), (x), (vii) and (iv) are just indeterminate. By implementing them into the neutrality assumptions in \eqref{mtox.1} and  \eqref{mtox.2}, they simply don't represent any usable result. We can therefor exclude them.

\subsection{Which is plausible?}
After canceling out 12 out of the 16 possible combinations, we should now think about which one might be the most plausible. Going after the public opinion, option (i) is the favorite. It is the only setting that is fully consistent with Keynes view on prices and wages, while assuming the origin currency pricing that is also kind of the gold standard of pricing assumptions. It is therefor not surprising that the public opinion tend to lean towards this scenario. But of course the other appreciation assumption is also quite robust. After \cite{krugman1986pricing} developed the theory of pricing-to-market and as we will see in \cite{fendel2008local} that this strategy is actually pretty promising for firms, one can assume that a majority of companies would actually adopt it. In combination with sticky prices and wages, it also leads to an appreciation.

The question for the empirical discussion will be: 
1. Is an economy tending more towards PCP or LCP?
2. Is the theory of sticky prices and wages in the short run robust?

It seems as we can we can compensate the non existing literature on our topic, by combining existing literature on the two questions above. 
\subsubsection{Buiter}

But there are some problems with the depreciation scenarios. Both require a change in prices and wages. In addition, the change will be asymmetric, meaning while prices and wages go up in the US, they will fall in the EU.

Asymmetric price Keynes?

\subsection{DCP?}
There is new theory on pricing behavior as developed in \cite{casas2017dominant}. It states, that there is neither destination currency pricing nor origin currency pricing, but dominant currecny pricing. Meaning, prices tend to stay sticky in the currency dominant currency.  


\subsubsection{Meins}
R. Fendel et. al.

Use his analysis to make clear what pricing assumption is the most plausible and 

''their foreign sales market are identified as those applying LCP. Our resultssuggest that the majority of the German exporting firms (70\%) apply LCP.They seem to compete more intensely with firms in foreign markets thanfirms that apply PCP. The majority of the total turnover of LCP firms stemsfrom foreign sales markets, whereas PCP firms generate only supplementaryshares of their turnover in export markets.In trying to identify differences between the two groups of firms, we findthat the question of PCP versus LCP seems to be related to the destination ofexports. PCP firms are more represented in countries of the euro area. Lookingat the geographical characteristics of export destinations excluding the euroarea, we find that PCP firms are more present in North America and EasternEurope, whereas LCP firms are relatively more active in Africa and Asia. Thisobservation might be driven by different degrees of market segmentation.PCP behavior applies to regions with more integrated markets. There is alsosome indication that LCP firms mainly compete with other exporters thatserve the same market, whereas PCP firms compete with local firms in theirexport markets.''
