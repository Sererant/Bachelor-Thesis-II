\section{Appendix}

\subsection{VAT = CPT}
To proof the equivalence of a value added tax, it is necessary to highlight the variable that differ from those in the corporate profit tax. Let us first understand what the difference is. While corporate profit does not tax U.S. wages, the VAT actually does. So effectively, there must be a difference in the after-tax profits, the tax revenues and the U.S. real product wages. 

We want to show that a border adjustment following a destination based cash flow tax is equivalent to one following a value added tax. This would be given, if the border tax adjustment neutrality condition are the same in both cases. 

Firstly, the after tax profit is defined as: 

\subsubsection*{Origin}
\begin{equation}
\begin{aligned}
\overline \Pi = \frac{\overline P_d}{1+\theta}\overline Q_d + \frac{\overline P_x}{1+\theta}\overline X - \overline e \overline P^*_m \overline M - \overline w  \left( \overline L_d + \overline L_x \right)  
\end{aligned}
\end{equation}

Starting with the origin case, we already see the difference.  While the origin based cash flow tax does not tax the U.S. payroll in (3), the value added tax does.

\subsubsection*{Destination}
\begin{equation}
\begin{aligned}
\underline \Pi = \frac{\underline P_d}{1+\theta}\underline Q_d +\underline P_x\underline X -  \left( 1+\theta \right) \underline e \underline P^*_m \underline M - \underline w  \left( \underline L_d + \underline L_x \right) 
\end{aligned}
\end{equation}

\subsubsection*{Difference}
\begin{equation}
\begin{aligned}
\underline \Pi - \overline \Pi = & \left( \frac{\underline P_d}{1+\theta}\underline Q_d +\underline P_x\underline X -  \left( 1+\theta \right) \underline e \underline P^*_m \underline M -  \underline w  \left( \underline L_d + \underline L_x \right)  \right)  \\ &-  \left( \frac{\overline P_d}{1+\theta}\overline Q_d + \frac{\overline P_x}{1+\theta}\overline X - \overline e \overline P^*_m \overline M - \overline w  \left( \overline L_d + \overline L_x \right)  \right) \\ 
= &\frac{1}{1+\theta}  \left( \underline P_d \underline Q_d - \overline P_d  \overline Q_d \right)  +\underline P_x\underline X - \frac{1}{1+\theta}\overline P_x \overline X \\ &-  \left(  \left( 1 + \theta \right) \underline e \underline P^*_m \underline M - \overline e \overline P^*_m \overline M \right)  -  \left( \underline w \underline L_d - \overline w \overline L_d + \underline w \underline L_x - \overline w \overline L_x \right) 
\end{aligned}
\end{equation}

\subsubsection*{Difference in real terms}
\begin{equation}
\begin{aligned}
\frac{\underline \Pi}{\underline P_d} - \frac{\overline \Pi}{\overline P_d} = &\frac{1}{1+\theta} \left( \underline Q_d - \overline Q_d \right)  + \frac{\underline P_x}{\overline P_d}\underline X -  \left( \frac{1}{1+\theta} \right) \frac{\overline P_x}{\overline P_d}\overline X \\ &-  \left( \frac{ \left( 1+\theta \right) \underline e \underline P^*_m}{\underline P_d}\underline M - \frac{\overline e \overline P^*_m}{\overline P_d}\overline M \right)  \\ &-  \left( \frac {\underline w}{\underline P_d} \underline L_d - \frac {\overline w}{\overline P_d} \overline L_d + \frac {\underline w}{\underline P_d} \underline L_x - \frac {\overline w}{\overline P_d} \overline L_x \right) 
\end{aligned}
\end{equation}

\subsubsection*{Conditions}

Output
\begin{equation}
\begin{aligned}
\underline Q_d = \overline Q_d
\end{aligned}
\end{equation}
\newline
\noindent Imports
\begin{equation}
\begin{aligned}
\underline M &= \overline M \\
\Rightarrow \frac{\underline P_m}{\underline P_d} &= \frac{\overline P_m}{\overline P_d} \quad or \quad \frac{\underline e \underline P_m^*}{\underline P_d} =  \left( \frac{1}{1+\theta} \right) \frac{\overline e \overline P^*_m}{\overline P_d}
\end{aligned}
\end{equation}
\noindent Exports
\begin{equation}
\begin{aligned}
\underline X &= \overline X \\ \frac{\underline P_x}{\underline P_d} &=  \left( \frac{1}{1+\theta} \right) \frac{\overline P_x}{\overline P_d} \quad or \quad \frac{\underline e \underline P_x^*}{\underline P_d} = \frac{\overline e \overline P^*_x}{\overline P_d}
\end{aligned}
\end{equation}
\noindent Labour
\begin{equation}
\begin{aligned}
\underline L_d &= \overline L_d, \underline L_x = \overline L_x \quad and \quad \frac{\underline w}{\underline P_d} = \frac{\overline w}{\overline P_d}
\end{aligned}
\end{equation}

Despite the fact, that there are differences regarding taxing the payroll in the origin and the destination case, the neutrality conditions are identical to the ones of the destination based cash flow tax. Let us have a look at the tax revenue. 

\subsubsection{Tax revenue}
\subsubsection*{Origin} 
\begin{equation} 
\begin{aligned}
\overline T = \frac{\theta}{1+\theta} \left( \overline Q_d \overline P_d + \overline X \overline P_x \right) 
\end{aligned} 
\end{equation}

Also the tax revenue is affected. Other then in (11), the taxed payroll is not deducted from the revenue itself.

\subsubsection*{Destination}
\begin{equation} 
\begin{aligned}
\underline T = \frac{\theta}{1+\theta} \left( \underline Q_d \underline P_d + \underline M  \left( 1 + \theta \right)  \underline e \underline P^*_m \right)
\end{aligned} 
\end{equation}

\subsubsection*{Difference} 
\begin{equation} 
\begin{aligned}
\underline T - \overline T = & \left( \frac{\theta}{1+\theta} \left( \underline Q_d \underline P_d + \underline M  \left( 1 + \theta \right)  \underline e \underline P^*_m \right)  - \theta \underline w  \left( \underline L_d + \underline L_x \right)  \right) \\ 
&= \frac{\theta}{1+\theta} \left( \underline P_d \underline Q_d - \overline P_d  \overline Q_d + \left( 1 + \theta \right) \underline e \underline P^*_m \underline M - \overline P_x \overline X \right) 
\end{aligned} 
\end{equation}

\subsubsection*{Difference in real terms}

As in (14) and (15) the neutrality condition of the tax return is the difference in real terms. 


\begin{equation} 
\begin{aligned}
    \frac{\underline T}{\underline P_d} - \frac{\overline T}{\overline P_d} &=  \frac{\theta}{1+\theta}  \left(  \underline Q_d - \overline Q_d +\frac{ \left( 1+\theta \right) \underline e \underline P^*_m}{\underline P_d}\underline M - \frac{\overline P_x}{\overline P_d} \overline X \right) \\
\end{aligned} 
\end{equation}


\subsection{Conditions}

Noch nicht sicher, ob und wie das drinnen bleibt. 
Quelle: \url{https://voxeu.org/article/border-adjustment-tax} and \cite{barbiero2018macroeconomics}

So before we assume that border tax adjustment neutrality will hold, we should have a look at the conditions under which it prevails. In general they can be separated into conditions sufficient for neutrality in the short run and in the long run. \\
1. Prices are sticky and the exchange rate adjusts \\
For the neutrality condition to hold, it is required that affected currencies have flexible exchange rates. We also assume that the related pass-through of taxes and exchange rate adjustments is symmetric between the countries.\footnote{A pass-through in context of exchange rates is basically the elasticity of import prices in respect to the price of foreign currency, often expressed in percent. So if the pass-through rate is 0.5, it means that if the dollar appreciates by 10\%, the import prices would only get 5\% cheaper instead of 10\%. We will discuss pass-throughs later, in the context of the law of one price and the discussion section.} Further, we assume that some prices are sticky and because, as stated above, we do not know what prices exactly, we will hold all combinations of import and export prices constant to examine possible effects. \\
2. There are no monetary policies that try to tackle exchange rate adjustments \\
We assume that if there are monetary policies implemented following an introduction of a border tax adjustment, they may concentrate on inflation and the output gap, but never the exchange rate itself. \footnote{This condition is mainly for the U.S.. While the U.S. federal reserve bank has the task to control inflation and support the domestic economy, the European Central Bank is allowed to focus solely on inflation.} As stated before, depreciation measurements were used by countries to boost their national economy and reduce their trade deficit. But since a natural exchange rate adjustment is needed to maintain border tax adjustment neutrality, the effects of a monetary policy may be distorting.
A failure of this two condition would lead to a breakdown of border tax adjustment neutrality in the short run, but it would unlikely be relevant for the long run. Following the Keynesian persuasion, in the long run, monetary policies are not relevant while prices and wages are flexible. This leads us back to the Lerner Symmetry. \\
3. \\ 'neutrality requires border taxes to be uniform and to cover all goods and services. Service sectors, such as tourism, whose sales to foreigners take place within borders, are not treated the same as exports that cross borders, which in turn affects neutrality.' \\
4. \\ 'the implementation of the BAT must take the form of a one- time, permanent, and unanticipated policy shift for it to be neutral. Other- wise, expectations of a border tax in the future will cause immediate exchange rate appreciations that affect the portfolio choices of private agents, and therefore will have real consequences. Similarly, neutrality fails to hold if the policy is expected to be reversed and is therefore transitory, or if the other countries are expected to retaliate with their own policies in the future.' 
The policy change must be unexpected, one-time and permanent. Otherwise, the dollar will appreciate, at least in part, in expectation before the policy is implemented, resulting in both distortions to international trade and to the portfolio choices of private agents. Similarly, neutrality would not hold if the policy is not expected to be permanently in place, or if the other countries are expected to retaliate with their own policies in the future.\\
5. \\ when trade is not balanced, neutrality continues to hold as long as all international assets and liabilities are denominated in foreign currency. If, however, some international holdings are denominated in domestic currency, then neutrality is no longer preserved. Because foreign assets held by the United States are mostly denominated in foreign currency, while its foreign liabilities are almost entirely in dollars, this generates a large, one-time transfer to the rest of the world and a capital loss for the United States of about 10 percent of U.S. annual GDP (Farhi, Gopinath, and Itskhoki 2017). Barbiero and others (2017), however, show that because this transfer is a small fraction of U.S. wealth, the real impact is quantitatively relatively small. \\
The latter three condition are essential for the border tax adjustment neutrality in short run AND in the long run. If they are not met, all models that are assuming neutrality will state wrong results. I will use section 4 to discuss their applicability and give an overview on empirical studies on this topic so far. \\

