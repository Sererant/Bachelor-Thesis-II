\section{Appendix}

\subsection{VAT = CPT}
To proof the equivalence of a value added tax, it is necessary to highlight the variable that differ from those in the corporate profit tax. Let us first understand what the difference is. While corporate profit does not tax U.S. wages, the VAT actually does. So effectively, there must be a difference in the after-tax profits, the tax revenues and the U.S. real product wages. 

We want to show that a border adjustment following a destination based cash flow tax is equivalent to one following a value added tax. This would be given, if the border tax adjustment neutrality condition are the same in both cases. 

Firstly, the after tax profit is defined as: 

\subsubsection*{Origin}
\begin{equation}
\begin{aligned}
\overline \Pi = \frac{\overline P_d}{1+\theta}\overline Q_d + \frac{\overline P_x}{1+\theta}\overline X - \overline e \overline P^*_m \overline M - \overline w  \left( \overline L_d + \overline L_x \right)  
\end{aligned}
\end{equation}

Starting with the origin case, we already see the difference.  While the origin based cash flow tax does not tax the U.S. payroll in (3), the value added tax does.

\subsubsection*{Destination}
\begin{equation}
\begin{aligned}
\underline \Pi = \frac{\underline P_d}{1+\theta}\underline Q_d +\underline P_x\underline X -  \left( 1+\theta \right) \underline e \underline P^*_m \underline M - \underline w  \left( \underline L_d + \underline L_x \right) 
\end{aligned}
\end{equation}

\subsubsection*{Difference}
\begin{equation}
\begin{aligned}
\underline \Pi - \overline \Pi = & \left( \frac{\underline P_d}{1+\theta}\underline Q_d +\underline P_x\underline X -  \left( 1+\theta \right) \underline e \underline P^*_m \underline M -  \underline w  \left( \underline L_d + \underline L_x \right)  \right)  \\ &-  \left( \frac{\overline P_d}{1+\theta}\overline Q_d + \frac{\overline P_x}{1+\theta}\overline X - \overline e \overline P^*_m \overline M - \overline w  \left( \overline L_d + \overline L_x \right)  \right) \\ 
= &\frac{1}{1+\theta}  \left( \underline P_d \underline Q_d - \overline P_d  \overline Q_d \right)  +\underline P_x\underline X - \frac{1}{1+\theta}\overline P_x \overline X \\ &-  \left(  \left( 1 + \theta \right) \underline e \underline P^*_m \underline M - \overline e \overline P^*_m \overline M \right)  -  \left( \underline w \underline L_d - \overline w \overline L_d + \underline w \underline L_x - \overline w \overline L_x \right) 
\end{aligned}
\end{equation}

\subsubsection*{Difference in real terms}
\begin{equation}
\begin{aligned}
\frac{\underline \Pi}{\underline P_d} - \frac{\overline \Pi}{\overline P_d} = &\frac{1}{1+\theta} \left( \underline Q_d - \overline Q_d \right)  + \frac{\underline P_x}{\overline P_d}\underline X -  \left( \frac{1}{1+\theta} \right) \frac{\overline P_x}{\overline P_d}\overline X \\ &-  \left( \frac{ \left( 1+\theta \right) \underline e \underline P^*_m}{\underline P_d}\underline M - \frac{\overline e \overline P^*_m}{\overline P_d}\overline M \right)  \\ &-  \left( \frac {\underline w}{\underline P_d} \underline L_d - \frac {\overline w}{\overline P_d} \overline L_d + \frac {\underline w}{\underline P_d} \underline L_x - \frac {\overline w}{\overline P_d} \overline L_x \right) 
\end{aligned}
\end{equation}

\subsubsection*{Conditions}

Output
\begin{equation}
\begin{aligned}
\underline Q_d = \overline Q_d
\end{aligned}
\end{equation}
\newline
\noindent Imports
\begin{equation}
\begin{aligned}
\underline M &= \overline M \\
\Rightarrow \frac{\underline P_m}{\underline P_d} &= \frac{\overline P_m}{\overline P_d} \quad or \quad \frac{\underline e \underline P_m^*}{\underline P_d} =  \left( \frac{1}{1+\theta} \right) \frac{\overline e \overline P^*_m}{\overline P_d}
\end{aligned}
\end{equation}
\noindent Exports
\begin{equation}
\begin{aligned}
\underline X &= \overline X \\ \frac{\underline P_x}{\underline P_d} &=  \left( \frac{1}{1+\theta} \right) \frac{\overline P_x}{\overline P_d} \quad or \quad \frac{\underline e \underline P_x^*}{\underline P_d} = \frac{\overline e \overline P^*_x}{\overline P_d}
\end{aligned}
\end{equation}
\noindent Labour
\begin{equation}
\begin{aligned}
\underline L_d &= \overline L_d, \underline L_x = \overline L_x \quad and \quad \frac{\underline w}{\underline P_d} = \frac{\overline w}{\overline P_d}
\end{aligned}
\end{equation}

Despite the fact, that there are differences regarding taxing the payroll in the origin and the destination case, the neutrality conditions are identical to the ones of the destination based cash flow tax. Let us have a look at the tax revenue. 

\subsubsection{Tax revenue}
\subsubsection*{Origin} 
\begin{equation} 
\begin{aligned}
\overline T = \frac{\theta}{1+\theta} \left( \overline Q_d \overline P_d + \overline X \overline P_x \right) 
\end{aligned} 
\end{equation}

Also the tax revenue is affected. Other then in (11), the taxed payroll is not deducted from the revenue itself.

\subsubsection*{Destination}
\begin{equation} 
\begin{aligned}
\underline T = \frac{\theta}{1+\theta} \left( \underline Q_d \underline P_d + \underline M  \left( 1 + \theta \right)  \underline e \underline P^*_m \right)
\end{aligned} 
\end{equation}

\subsubsection*{Difference} 
\begin{equation} 
\begin{aligned}
\underline T - \overline T = & \left( \frac{\theta}{1+\theta} \left( \underline Q_d \underline P_d + \underline M  \left( 1 + \theta \right)  \underline e \underline P^*_m \right)  - \theta \underline w  \left( \underline L_d + \underline L_x \right)  \right) \\ 
&= \frac{\theta}{1+\theta} \left( \underline P_d \underline Q_d - \overline P_d  \overline Q_d + \left( 1 + \theta \right) \underline e \underline P^*_m \underline M - \overline P_x \overline X \right) 
\end{aligned} 
\end{equation}

\subsubsection*{Difference in real terms}

As in (14) and (15) the neutrality condition of the tax return is the difference in real terms. 


\begin{equation} 
\begin{aligned}
    \frac{\underline T}{\underline P_d} - \frac{\overline T}{\overline P_d} &=  \frac{\theta}{1+\theta}  \left(  \underline Q_d - \overline Q_d +\frac{ \left( 1+\theta \right) \underline e \underline P^*_m}{\underline P_d}\underline M - \frac{\overline P_x}{\overline P_d} \overline X \right) \\
\end{aligned} 
\end{equation}
