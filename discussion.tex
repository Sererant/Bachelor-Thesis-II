\section{Discussion} \label{discussion}
\subsection{Critic}
The central assumption of \cite{buiter2017exchange} is the existence of nominal price rigidities and even for strong Keynesian believers, this is only relevant for the short run. In the long run, as for example described by the Phillips Curve, the prices would find their new equilibrium. But on the other hand, we know that Lerner Symmetry is meant primarily to predict the long run. And as pointed out earlier, the conditions for it to prevail in the short run are hardly realistic. 



\subsection{Empirical Evidence}
\subsection{Balance of Trade}
What will happen to the balance of trade?
The Marshall-Lerner 
\subsection{Exchange-rate pass-through}
So following \cite{gopinath2017macroeconomic} a stickiness of prices and wages in producers currency lead to a 100\% pass-through of exchange rates and taxes into prices and lead to an appreciation of the dollar (If the tax is imposed by the U.S. of course). On the other hand, if the prices and wages are sticky in the local currency, the pass-through is 0\% and there is no appreciation. This is an interesting result. Though it is not exactly clear, we can assume and given our analysis we can be sure, that he is describing the tax-exclusive prices to be sticky. Following table 1, (i) shows his first stickiness assumption. Imports in Euro and exports in Dollar reflect a producer pricing. As well as Gopinath, we see an appreciation of the Dollar. More interesting is his second case. He states, that with prices sticky in the local currency, there is a 0\% pass-through and no appreciation needed to prevail neutrality. For local prices, or consumer prices (tax-exclusive), our case is (xi) where imports are sticky in dollar and exports are sticky in euro. We concluded a depreciation of the currency. At first, this might sound contrary. But considering the Keynes statement that we cited in the literature section and that he gave to the finance committee in 1931, a border tax adjustment mimics the effects of a depreciation when the currency is fixed under a gold standard.  So eventually, the same neutrality effects can be achived by ...

\subsection{financial chaos}
\subsection{Simple Language}
\url{https://taxfoundation.org/exchange-rates-and-border-adjustment/#_ftnref1}

\subsection{Will the BTA neutrality hold?}
The proof as made by Feldstein and Krugman (1990) on inter-temporal (balanced) trade and imperfectly markets by Ray (1975), was conducted in the neoclassical fashion. Two countries, two final goods, no trade costs and no distortionary taxes. Since this was a good approximation for an economy in the mid of the last century, its hardly usable for a precise analysis today. There are multiple layered trade costs, multinational firms, imperfect competition and increasing returns. While Costinot and Werning (2018) are testing the Lerner Symmetry on its robustness in a modern setting, Barbiero et. al. (2017) analysis the border adjustment tax with a New-Keynsian DSGE model. 

Desai and Hines (2005) found some evidence from 136 countries in 2000 that countries with a VAT actually trade less. They also showed the same effects for an unbalanced panel of 168 countries between 1950 and 2000. In their view, traded goods are subject to higher VAT rates and therefor consumption and production of non-traded goods is higher. Similar effects were found by Nicholson (2010) with panel data over 29 industries, 146 countries and over 12 years, thus here the effects on export side is more significant then on the import side, that is dominated by fossil fuels and materials (besseres Wort finden). 

\subsection{Is the Lerner Symmetry transitive?}
One of the fundamental pillars of Buiters paper is the assumption, that the Symmetry of import and export taxes as described in Lerner (1936) follows the rule of transitivity. So that if import taxes are equal to export taxes, it follows that import subsidies are equal to export subsidies and therefor an equal import tax and export subsidy cancels each other out. Casas (1991) is arguing against this statement. While he agrees with Lerner (1936) that import and export taxes effect consumption, production and domestic prices in the same way, he shows that an export subsidy generates a production loss and an import subsidy generates a consumption loss. Should this be true, it would invalidate our model since our basic assumption of border tax adjustment neutrality would not hold anymore. Especially in the short run it would not be able for a economy to compensate losses 

\subsection{Compliance of the BTA concerning the WTO}
One issue that potentially prevents the introduction of a destination-based tax regime coupled with an uniform cash flow CPT is its discriminatory character which doesn't comply with the equity standards of the World Trade Organization (WTO). Under such a regime, the domestic producer wouldn't pay the full value equivalent in taxes, but only the fraction concerning the production of the good. It can thus offer the good at a cheaper price than competitors abroad and gains a comparative advantage. 

However, this hinges critically on the assumption of BTA neutrality as discussed in section \textit{?}. If BTA neutrality holds, the price advantage with a uniform rate cash flow CPT diminishes ,given that the real part of the foreign producers payroll diminishes by exactly the amount that would belong to the domestic producer. 

Especially the protectionist parties are challenging WTO focus on import taxes rather than export taxes. If they are both neutral, why focusing only on one part of it? After Barbiero et. al (2018)(anderes Paper) stressed the Lerner Symmetry with a full DSGE Model they came to the conclusion, that considering many more variables then Lerner did, the symmetry is not really symmetric at all. They found out, that a tax on import distorts trade more than a tax on exports. So the WTO might be not wrong with their focus after all.
