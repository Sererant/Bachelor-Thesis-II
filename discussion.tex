\section{Discussion} \label{discussion}
I will use the discussion section to present empirical evidence concerning price constanccy assumptions and the border tax adjustment neutrality. Afterwards, I will look at theoretical work that criticizes the Lerner Symmetry, express some critic on the model as conducted by \cite{buiter} and in the end give an outlook on how an implementation would look like in reality.


\subsection{Empirical evidence}
After \cite{krugman1986pricing} developed the theory of pricing to market, there were numerous attempts to prove empirically which assumption is the most plausible. Yet, the outcomes are mixed. \cite{fendel2008local} took survey data from German exporters to identify if they apply origin or destination currency pricing. They concluded that 70\% applied destination currency pricing, with strong variation regarding the destination of the exports. While exports to Asia and Africa were mainly subject to origin currency pricing, exports to North America tended more to apply destination currency pricing. This outcome actually leads to a new pricing approach that would replace the two previous ones and was developed by \cite{casas2017dominant}. It states, that there is neither destination currency pricing nor origin currency pricing, but dominant currency pricing. Meaning, prices tend to stay sticky in the dominant currency. This would suit the result of \cite{fendel2008local}. If I compare the Dollar to the Euro, the Dollar is more dominant as a currency. So prices tend to stay sticky in terms of the Dollar, what supports the destination currency pricing behavior of German exporters.%\footnote{It is especially valuable because \cite{fendel2008local} delivers his empirical study for real pricing and not just invoicing as in }
Unfortunately, our model does not indicate any exchange rate movements for this case, neither tax-inclusive nor tax-exclusive. The reason is that (ix), (iii), (viii) and (xiv) are inconsistent with our neutrality assumptions. Even though the empirical research might suggests that dominant currency pricing is the most plausible pricing assumption, it still only presents a share in the general pricing behavior. So there should be theoretical and empirical work on how the presented model would react if there is not one but three pricing behaviors simultaneously. While pricing assumptions are a well studied topic, the question whether prices are sticky tax-inclusive or tax-exclusive is poorly understood. As \cite{buiter2017exchange}, I was not able to find empirical studies addressing this question directly. But one can approximate behavior regarding tax-inclusive and tax-exclusive pricing through a look on the exchange rate path-through.
So following \cite{gopinath2017macroeconomic} a stickiness of prices and wages in the producers currency lead to a 100\% pass-through of exchange rates and taxes into prices and lead to an appreciation of the dollar (If the tax is imposed by the U.S. of course). On the other hand, if the prices and wages are sticky in the local currency, the pass-through is 0\% and there is no appreciation. This is an interesting result. Though it is not exactly clear, we can assume and given our analysis we can be sure, that he is describing the tax-exclusive prices to be sticky. Following table 1, (i) shows her first stickiness assumption. Imports in Euro and exports in Dollar reflect a producer pricing. As well as \cite{gopinath2017macroeconomic}, we see an appreciation of the Dollar. More interesting is her second case. She states, that with prices sticky in the local currency, there is a 0\% pass-through and no appreciation needed to prevail neutrality. For local prices, or consumer prices (tax-exclusive), our case is (xi) where imports are sticky in dollar and exports are sticky in euro. I concluded a depreciation of the currency. At first, this might sound contrary. But considering J.M. Keynes statement that I cited in the literature section and that he gave to the finance committee in 1931, a border tax adjustment mimics the effects of a depreciation when the currency is fixed under a gold standard. One might also take a look into empirical studies made solely on tariff pass-through. For example \cite{feenstra1989symmetric} studied pass-through of tariffs levied on Japanese car imports in the U.S. and \cite{ludema2016tariff} also studied pass-through of tariffs, but in the context of firms' exports that were subject to foreign import taxes. Both concluded that tax-exclusive prices tend to change and absorb the shocks created by the tariffs, while tax-inclusive prices stay mostly constant. However, even though the pass-through approach might be interesting, the empirical studies fail to investigate both of our constancy dimensions at once. They might be examining the second dimension of tax-inclusive and tax-exclusive prices, but they do not consider origin or destination currency pricing behavior. So in the end, there is yet no answer on which price constancy assumption and consequently which exchange rate movement is the most plausible. 

After presenting empirical evidence concerning the two dimensions of price constancy, it is also worth looking at studies conducted to proof our central assumption of border tax adjustment neutrality.  

\cite{hines2005value} found some evidence from 136 countries in 2000 that countries with a VAT actually trade less. They also showed the same effects for an unbalanced panel of 168 countries between 1950 and 2000. In their view, traded goods are subject to higher VAT rates and therefor consumption and production of non-traded goods is higher. Similar effects were found by \cite{nicholson2010value} with panel data over 29 industries, 146 countries and over 12 years, thus here the effects on the export side is more significant then on the import side, that is dominated by fossil fuels and minerals. 
The reason for the inaccuracy of the earlier theories is the scope of their models. The proof for border tax adjustment neutrality as made by \cite{Feldstein&Krugman} on inter-temporal (balanced) trade and imperfectly markets by \cite{ray1975impact}, was conducted in the neoclassical fashion. Two countries, two final goods, no trade costs and no distortionary taxes. Since this was a good approximation for an economy in the mid of the last century, its hardly usable for a precise analysis today. There are multiple layered trade costs, multinational firms, imperfect competition and increasing returns. \cite{linde2017macroeconomic} analyzed the effects of a border tax adjustment with a New-Keynesian DSGE model.\footnote{The dynamic stochastic general equilibrium model is a empirical validated method to explain and predict macroeconomic effects.} They found support for the Lerner Symmetry and the neutrality of import taxes and export subsidies in the long run, even under incomplete markets. Whereas the short run depends on the scale and speed of the exchange rate adjustment. But the potential negative effects are rather small and mainly visible as a fall in foreign output. They actually had another interesting outcome. After picking up the idea of dominant currency pricing, like \cite{casas2017dominant} did, they found out that this is not breaking the neutrality conditions. Interestingly, this contradicts with the results of my model, since, as mentioned above, the dominant pricing combinations in table 1 indicate an inconsistency with the border tax adjustment neutrality.~\\

Apart from empirical evidence, there is also theoretical work that questions the border tax adjustment neutrality, especially the underlying Lerner Symmetry.

From \cite{grossman1980border} to \cite{buiter2017exchange} there is the common perception that the symmetry of import and export taxes, as described in \cite{lerner1936symmetry}, follows the rule of transitivity. So that if import taxes are equal to export taxes, it follows that import subsidies are equal to export subsidies and therefor an equal import tax and export subsidy cancels each other out. \cite{casas1991lerner} is arguing against this statement. While he agrees with \cite{lerner1936symmetry} that import and export taxes effect consumption, production and real domestic prices in the same way, he shows that an export subsidy generates a production loss and an import subsidy generates a consumption loss. Especially in the short run it would not be possible for an economy to compensate losses generated through this asymmetry and therefore create distortions that could break the border adjustment neutrality. 

\subsubsection{Critic on Buiters Model}

One critic of mine on \cite{buiter2017exchange} concerns exactly this. Beginning with the first constant pricing assumption dimension, he only pays attention to origin currency pricing and destination currency pricing, but neglects other possible pricing behaviors. The most notable here is the dominant pricing assumption. Following \cite{casas2017dominant} it is not only the most plausible, with the findings of \cite{linde2017macroeconomic} it is also proven not to break the border tax adjustment neutrality. This is in fact contrary to our earlier findings, where according to table 1 they are inconsistent with our neutrality conditions. It should be addressed in future studies on this topic. ~\\
Going back to the more general justification of the border tax adjustment neutrality, \cite{barbiero2018macroeconomics} defined five necessary condition. While I explain them more thoroughly in the appendix, it is worth here to sum up his finding again. He concludes that the neutrality conditions are highly unlikely to be met, in the context of a destination based corporate profit tax and interestingly even more with the value-added tax. This is another critic of mine on the model provided by \cite{buiter2017exchange}. He assumes that value-added taxes and corporate profit taxes are comparable in the context of border tax adjustments. After looking at the results of \cite{barbiero2018macroeconomics} this seems also questionable, since the effects of border tax adjustments are different. So after discussing the central assumptions of the border adjustment tax neutrality and how they might brake, one should also take a look at political and economic motivation behind an introduction of a destination based corporate profit tax and assess their validity.

\subsubsection{What should be done in the future}
1. Dominant pricing assumption
2. VAT and CPT are equal 
3. Include capital expenditures and capital ...

%Another arguable aspect is the assumptions of price rigidites in general. Because even for strong Keynesian believers, this is only relevant for the short run. In the long run, as for example described by the Phillips Curve, the prices would find their new equilibrium. But on the other hand, we know that Lerner Symmetry is meant primarily to predict the long run. And as pointed out earlier, the conditions for it to prevail in the short run are hardly realistic. 

\subsection{The political di}
Following 

- Auerbach and his intentions 
- The intentions of the republican party 
- WTO and potential costs 
- Financial chaos because of uncertainty 

Originally the 



\subsection{Compliance of the BTA concerning the WTO}
One issue that potentially prevents the introduction of a destination-based tax regime coupled with an uniform corporate profit tax is its discriminatory character which does not comply with the equity standards of the World Trade Organization. Under such a regime, the domestic producer wouldn't pay the full value equivalent in taxes, but only the fraction concerning the production of the good. It can thus offer the good at a cheaper price than competitors abroad and gains a comparative advantage. 

However, this translates critically on the assumption of border tax adjustment neutrality as discussed previously. If border tax adjustment neutrality holds, the price advantage with a uniform rate destination based corporate profit tax shrinks, given that the real part of the foreign producers payroll falls by exactly the amount that would belong to the domestic producer. 

Especially the protectionist parties are challenging the World Trading Organization focus on import taxes rather than export taxes. If they are both neutral, why focusing only on one part of it? After \cite{barbiero2018macroeconomics} stressed the Lerner Symmetry with a full DSGE Model they came to the conclusion, that considering many more variables then Lerner did, the symmetry is not really symmetric at all. They found out, that a tax on import distorts trade more than a tax on exports. So the World Trading Organization might be not wrong with their focus after all.

\subsection{What could go wrong?}
In the end 